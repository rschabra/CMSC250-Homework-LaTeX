%%%%%%%%%%%%%%%%%%%%%%%%%%%%%%%%%%%%%%%%%%%%%%%%%%%%%%
% LaTeX Template for UMD Assignments.
% Created by Jason Filippou (jasonfil@cs.umd.edu)
% Maintained at: https://github.com/JasonFil/UMDAssignmentTemplates
% Check LICENSE file on GitHub for licensing. 
%%%%%%%%%%%%%%%%%%%%%%%%%%%%%%%%%%%%%%%%%%%%%%%%%%%%%%%%%%%%%%%%%%%%%%%%%%%%%%%%%%%%%%%%%%

% Document class will always be extarticle for the purposes of UMD assignments
\documentclass[letterpaper,14pt]{extarticle}

%%%%%%%%%%%%%  IMPORT MACRO FILES AS NEEDED %%%%%%%%%%%
\usepackage{amsgen,amsmath,amstext,amsbsy,amsopn,amssymb,amsthm,stackengine}
\usepackage{breqn} % For breaking equations.
\usepackage[usenames,dvipsnames,svgnames,table]{xcolor}
\usepackage{array, nicefrac, mathtools}
\usepackage[bottom]{footmisc} % To keep the footnote at the bottom even after putting answering environments.
\usepackage{verbatim}
\usepackage{booktabs} % for multicolumn
\usepackage[font=normalsize,skip=0pt, justification=centering]{caption, subcaption}
\usepackage[colorlinks=true,linkcolor=blue,urlcolor=blue]{hyperref}
\usepackage{float,relsize,setspace,enumitem,pbox,cleveref,multicol,multirow}
\usepackage{censor}
\usepackage{textcomp} % for the cent symbol
\usepackage{multido}
\usepackage{bbding} % Has a checkmark symbol reachable through \Checkmark
\usepackage{tikz,mdframed}
\usepackage{epsdice} 

% Basic math
%\usepackage{amsgen,amsmath,amstext,amsbsy,amsopn,amssymb,amsthm}
%\usepackage[usenames,dvipsnames,svgnames,table]{xcolor}
%\usepackage{array, nicefrac, mathtools}

% Theorems, definitions, equations, lemmas
\newtheorem{thm}{Theorem}[section]
\newtheorem{prop}[thm]{Proposition}
\newtheorem{lem}[thm]{Lemma}
\newtheorem{cor}[thm]{Corollary}
\newtheorem{defn}{Definition}
\newtheorem{rem}[thm]{Remark}
\numberwithin{equation}{section}
\newtheorem*{defn*}{Definition} % Theorem environments with no numbering
\newtheorem*{prop*}{Proposition}
\newtheorem*{thm*}{Theorem}

% For negation and quantifiers in Discrete Math
\newcommand{\shortsim}{\raise.17ex\hbox{$\scriptstyle \sim$}}
\renewcommand{\neg}{\shortsim}
\renewcommand{\nexists}{\neg(\exists}
\newcommand{\nequiv}{\ensuremath{\not\equiv}}

\newcommand{\myline}[1]{\underline{\hspace{#1}}}
\newcounter{parts}
\newcounter{problems}[parts]
\newcounter{questions}[problems]
\newcounter{subquestions}[questions]
\newcommand{\hwpart}[1]{
	\stepcounter{parts}
	\noindent\makebox[\textwidth]{\LARGE \bf Part \arabic{parts} - #1}
	\\
}
\newcommand{\problem}[2]{\stepcounter{problems} {\Large \bf \noindent Problem \arabic{problems}: #1 \emph{(#2 pts)}} \\[.4cm]}
\newcommand{\question}[2]{\stepcounter{questions} \noindent{\emph{\Large Question (\alph{questions}): #1 (#2 pts) }\\[.4cm]}}
\newcommand{\subquestion}[2]{\stepcounter{subquestions} \noindent{\emph{\Large Subquestion (\roman{subquestions}): #1 (#2 pts) }\\[.4cm]}}

% Solution formatting
\newcommand{\solution}[1]{{\color{red}{#1}}}
% Some standard centering and italicization of text.
\newcommand{\frontrowcenter}[1]{\begin{center}{\em \Large  #1  }\end{center}}

% A blank page
\newcommand{\blankpage}{
\clearpage
\vspace*{\fill}
	\begin{minipage}{\textwidth}
		 \Large \textbf{THIS PAGE INTENTIONALLY LEFT BLANK}\\
	\end{minipage}
\vfill % equivalent to \vspace{\fill}
\clearpage
}

\newcommand{\answerspace}[1]{
	\begin{center}
		\textbf{BEGIN YOUR ANSWER BELOW THIS LINE} \\ \hrulefill \vspace{#1} \\ \hrulefill
	\end{center}
}

\newcommand{\answerspacefullpage}{
	\begin{center}
		\textbf{BEGIN YOUR ANSWER BELOW THIS LINE} \\ \hrulefill \pagebreak
	\end{center}
}

\newcommand{\additionalanswerspace}[1]{
	\begin{center}
		\textbf{CONTINUE YOUR ANSWER BELOW THIS LINE } \\ \hrulefill \vspace{#1} \\ \hrulefill
	\end{center}
}

\newcommand{\additionalanswerspacefullpage}{
	\begin{center}
		\textbf{CONTINUE YOUR ANSWER BELOW THIS LINE} \\ \hrulefill \pagebreak
	\end{center}
}

\newcommand{\freespace}[1]{
	\begin{center}
		\large \textbf{SCRAP SPACE BELOW} \\ 
		\hrulefill
		\pagebreak 
	\end{center}
}

% Centered line
\newcommand{\mycenterline}[1]{
	\begin{center}
		\myline{#1}
	\end{center}
}

% Space for T/F:
\newcommand{\tfline}{\myline{.5cm}}

% For quick parenthesized and italicized point annotation.
\newcommand{\pts}[1]{{\em (#1 pts)}}
\newcommand{\onept}{{\em (1 pt)}}

% \item environments coupled with a line at the end, for students to write T and F in.
\newcommand{\tfitem}[1]{\item #1 \null\hfill \framebox(25,25){} \\ \hdashrule{0.95\textwidth}{1pt}{2pt}}
\newcommand{\setitem}[1]{\tfitem{$\curlybraces{#1}$} }
\newcommand{\lineitem}[2]{\item #1 \null \hfill \myline{#2}}

% Some circles and squares for students to fill in.
\newcommand{\whitecircle}[1]{\tikz[baseline=-0.5ex]\draw[black, radius=#1] (0,0) circle ;}
\newcommand{\whitesquare}[1]{\tikz\draw[black] (0,0) rectangle(#1, #1) ;}

% Emphasis
\newcommand{\F}{$\mathbf{F}$}
\newcommand{\T}{$\mathbf{T}$}
\newcommand{\False}{\textbf{False}}
\newcommand{\false}{\textbf{false}}
\newcommand{\True}{\textbf{True}}
\newcommand{\true}{\textbf{true}}
\newcommand{\TODO}{\textcolor{red}{\textbf{TODO}}}
\newcommand{\TBD}{\textcolor{red}{\textbf{TBD}}}
\newcommand{\codeemph}[1]{\texttt{\textbf{#1}}}
\newcommand{\Red}{\textcolor{red}{Red}}
\newcommand{\red}{\textcolor{red}{red}}
\newcommand{\makered}[1]{\textcolor{red}{#1}}
\newcommand{\Rbbst}{\textcolor{red}{Red}-black tree}
\newcommand{\rbbst}{\textcolor{red}{red}-black tree}

\newcommand{\nullc}{\texttt{\textbackslash 0}}
\newcommand{\SPC}{\texttt{SPC}}
\newcommand{\checkmarkifsoln}{\ifshowsoln \textcolor{red}{\Checkmark} \fi}
\newcommand{\dialitem}[2]{\item[-] \textbf{#1}: ``\textit{#2}"} % For dialogue building.
% Repeating commands many times!
\newcommand{\Repeat}{\multido{\i=1+1}} 

%%%%%%%%%%%%%%%%%%%%%%%%%%%%%%%%%%%%%%%%%%%%%%%%%
%
%  STUDENTS - Below, write your name, UID, and section number in place of the relevant "\myline{}" command.
%
%%%%%%%%%%%%%%%%%%%%%%%%%%%%%%%%%%%%%%%%%%%%%%%%%

\newcommand{\homeworkdata}[6]{
\begin{mdframed}[linewidth=1pt]
    \noindent\makebox[\textwidth]{\LARGE \bf #1, #2 }
    \Repeat{2}{\\}
    \noindent\makebox[\textwidth]{\Large \bf  Homework \##3 }
      \Repeat{2}{\\}
       \noindent\makebox[\textwidth]{\large \bf  Due: #4 (on-time)}
    \Repeat{2}{\\}
    \noindent\makebox[\textwidth]{\large \bf \hspace{1.5in} #5 (late, #6\% of credit) }
    \Repeat{2}{\\}
    \noindent\makebox[\textwidth]{\large \bf  \textbf{\underline{First}} \& \textbf{\underline{Last}} Name: \enspace Rohan Chabra}
    \Repeat{2}{\\}
    \noindent\makebox[\textwidth]{\large \bf  UID (9 digits): \enspace 114466990}
    \Repeat{2}{\\}
    \noindent\makebox[\textwidth]{\large \bf Section number (4 digits): \enspace 0207}
    \\
\end{mdframed}
}

\usepackage[inner=1cm,outer=1cm,top=1cm,bottom=2cm]{geometry}
\setlength{\parindent}{2em}
\setlength{\itemindent}{.5in}

% Title of the current document
\title{CMSC250, Fall '20: Homework \#1}


%%%%%%%%%%%%%%%%%%%%%%%%%%%%%%%%%%%%%%%%%%%%%
%
%  STUDENTS - Your homework begins here.
%
%%%%%%%%%%%%%%%%%%%%%%%%%%%%%%%%%%%%%%%%%%%%%
\begin{document}
\homeworkdata{CMSC 250}{Fall 2020}{1}{Wednesday 09-09, 11:59pm}{Thursday 09-10, 11:59pm}{80}

\vspace{.2in} 
\hwpart{Material From Tuesday, 09-01}
\\

\problem{Converting Logic and English}{40}
\question{Logic to English}{20}
% \vspace{-.1in} 

Translate each propositional logic expression into an English sentence.\\
\indent Use the following logical symbols:
\begin{itemize}
	\item Let $p$ = "It is raining."
	\item Let $q$ = "It is cloudy."
	\item Let $r$ = "The weather is bad."
\end{itemize}

\begin{enumerate}[label = (\alph*)]
	\lineitem {$p \land q$} {0 in} It is raining and it is cloudy.
	\lineitem {$(\neg p) \lor (\neg q)$} {0 in} Either it is not raining or it is not cloudy.
	\lineitem {$(p \lor q) \Rightarrow r$} {0 in} If it is either raining or it is cloudy, then the weather is bad.
			\null \hfill
	\lineitem {$(\neg r) \Leftrightarrow ((\neg p) \land (\neg q))$}{.5 in}The weather is not bad, if and only if it is not raining and it is not cloudy.
			\null \hfill 
\end{enumerate}

\pagebreak

\question{English to Logic}{20}

Translate each English sentence into a propositional logic expression. Note that the \emph{truth value} of the statement is irrelevant - we want the direct translation, no matter whether the statement is true or false.\\

Use the following logical symbols:
\begin{itemize}
	\item Let $x$ = "Alice has ingredients."
	\item Let $y$ = "Alice has a recipe."
	\item Let $z$ = "Alice can bake a cake."
\end{itemize}

\begin{enumerate}[label = (\alph*)]
	\item {Alice has a recipe and ingredients.\\\\}
    {$y \land x$}  
	\item {Either Alice can bake a cake, or Alice does not have a recipe (or both).\\\\}
	{$z \lor (\neg y)$}
	\item {If Alice has ingredients and Alice has a recipe, then Alice can bake a cake.\\\\}
	{$(x \land y) \Rightarrow z$}
	\item {Alice can bake a cake, if and only if she has ingredients or does not have a recipe (or both).\\\\}
	{$z \Leftrightarrow (x \lor (\neg y))$}
\end{enumerate}

\pagebreak

\problem{Truth Tables}{60}
\question{A simple one}{30}

Fill-in the truth table below, which was built for the expression:

{\Large $$((\neg p) \land q) \lor (r \land p)$$ } 

\begin{center}
	\begin{table}[H] 
		\large 
		\setlength{\tabcolsep}{16pt}
%		\renewcommand{\arraystretch}{1.2}
		\begin{tabular}{|c|c|c|c|c|c|c| p{5.25in}|} \hline 
			$p$ & $q$ & $r$ & $\neg p$ & $(\neg p) \land q$ & $r \land p$ & $((\neg p) \land q) \lor (r \land p)$\\ \hline
			\F & \F & \F & \T & \F & \F & \F  \\ \hline 		
			\F & \F & \T & \T & \F & \F & \F \\ \hline 		
			\F & \T & \F & \T & \T & \F & \T  \\ \hline 		
			\F & \T & \T & \T & \T & \F & \T \\ \hline
			\T & \F & \F & \F & \F & \F & \F \\ \hline 		
			\T & \F & \T & \F & \F & \T & \T \\ \hline
			\T & \T & \F & \F & \F & \F & \F \\ \hline 		
			\T & \T & \T & \F & \F & \T & \T \\ \hline												
		\end{tabular}
	\end{table}
\end{center}
%

\begin{itemize}
	\lineitem {Is this expression a \textbf{tautology}, a \textbf{contradiction}, or a \textbf{contingency} (a statement that is neither a tautology nor a contradiction)?\\}{0 in} This expression is a \textbf{contingency}.
\end{itemize}

\pagebreak

\question{A less simple one}{30}

Perform the exact same task for the following truth table, built for the expression: 

{\Large $$[(p \land (\neg p)) \land q] \Rightarrow r$$} 


\begin{center}
	\begin{table}[H] 
		\large 
		\setlength{\tabcolsep}{16pt}
%		\renewcommand{\arraystretch}{1.2}
		\begin{tabular}{|c|c|c|c|c|c|c|} \hline 
			$p$ & $q$ & $r$ & $\neg p$ & $p \land (\neg p)$ & $(p \land (\neg p)) \land q$ & $[(p \land (\neg p)) \land q] \Rightarrow r$ \\
			\F & \F & \F & \T & \F & \F & \T  \\ \hline 		
			\F & \F & \T & \T & \F & \F & \T \\ \hline 		
			\F & \T & \F & \T & \F & \F & \T \\ \hline 		
			\F & \T & \T & \T & \F & \F & \T \\ \hline
			\T & \F & \F & \F & \T & \F & \T \\ \hline 		
			\T & \F & \T & \F & \T & \F & \T \\ \hline
			\T & \T & \F & \F & \F & \F & \T \\ \hline 		
			\T & \T & \T & \F & \F & \F & \T \\ \hline											
		\end{tabular}
	\end{table}
\end{center}
%

\begin{itemize}
	\lineitem {Is this expression a \textbf{tautology}, a \textbf{contradiction}, or a \textbf{contingency} (a statement that is neither a tautology nor a contradiction)?\\}{0 in} This expression is a \textbf{tautology}.
\end{itemize}

\pagebreak

\hwpart{Material From Thursday, 09-03}

\problem{Proving Logical Equivalence}{60}
\question{Using a Truth Table}{30}

Observe the following propositional logic expressions:

\vspace{.2in}

$$\mathlarger{ p \Rightarrow (q \lor r)}$$
$$\mathlarger{ \big (p \land (\neg q)\big ) \Rightarrow r }$$ 

\vspace{.3in}



Prove that the above expressions are \emph{logically equivalent} using a \textbf{truth table}.


\answerspacefullpage
\begin{center}
	\begin{table}[H] 
		\large 
		\setlength{\tabcolsep}{16pt}
%		\renewcommand{\arraystretch}{1.2}
		\begin{tabular}{|c|c|c|c|c|} \hline 
			$p$ & $q$ & $r$ & $q \lor r$ & $p \Rightarrow (q \lor r)$ \\
			\F & \F & \F & \F & \T \\ \hline 		
			\F & \F & \T & \T & \T \\ \hline 		
			\F & \T & \F & \T & \T \\ \hline 		
			\F & \T & \T & \T & \T \\ \hline
			\T & \F & \F & \F & \F \\ \hline 		
			\T & \F & \T & \T & \T \\ \hline
			\T & \T & \F & \T & \T \\ \hline 		
			\T & \T & \T & \T & \T \\ \hline											
		\end{tabular}
	\end{table}
\end{center}	
\hline
\vspace{.2 in}
\textbf{The final column of both tables are identical, therefore the two expressions are equivalent.}
\vspace{.2 in}
\hline
\begin{center}
	\begin{table}[H] 
		\large 
		\setlength{\tabcolsep}{16pt}
%		\renewcommand{\arraystretch}{1.2}
		\begin{tabular}{|c|c|c|c|c|c|} \hline 
			$p$ & $q$ & $r$ & $\neg q$ & $p \land (\neg q)$ & $(p \land (\neg q)) \Rightarrow r$\\
			\F & \F & \F & \T & \F & \T \\ \hline 		
			\F & \F & \T & \T & \F & \T \\ \hline 		
			\F & \T & \F & \F & \F & \T \\ \hline 		
			\F & \T & \T & \F & \F & \T \\ \hline
			\T & \F & \F & \T & \T & \F \\ \hline 		
			\T & \F & \T & \T & \T & \T \\ \hline
			\T & \T & \F & \F & \F & \T \\ \hline 		
			\T & \T & \T & \F & \T & \T \\ \hline											
		\end{tabular}
	\end{table}
\end{center}
\additionalanswerspacefullpage


\question{Using Epp's laws of Propositional Logic}{30}

Observe the following propositional logic expressions:

\vspace{.2in}

$$\mathlarger{\neg( (x \lor z) \Rightarrow y) \lor z}$$
$$\mathlarger{(x \lor z) \land ((\neg y) \lor z) }$$ 

\vspace{.3in}

Prove that the above expressions are \emph{logically equivalent} using the \textbf{laws of\\ propositional logic}. Note that, every time you apply a law of propositional logic, you need to document it on the right of the derivation. That way \textit{we} know that {\em you} know what law you are applying.

\answerspacefullpage
$$\mathlarger{\neg( (x \lor z) \Rightarrow y) \lor z}$$
\additionalanswerspacefullpage

\problem{Understanding Logical Expressions}{40}
\question{Working with conditionals}{15}

Consider the following English sentence:

\begin{center}
	\emph{If my laptop is plugged in, then it charges overnight.}
\end{center}

Write the \textbf{converse}, \textbf{inverse}, and \textbf{contrapositive} of this statement.

\begin{enumerate}[label=(\alph*)]
	\item Converse \\\\\\
	\item Inverse \\\\\\
	\item Contrapositive \\\\\\
\end{enumerate}
\pagebreak
\question{Simplifying}{25}
Using Epp's axioms of propositional logic equivalence, simplify the following compound statement \textbf{as much as possible}. For every derivation you make, write down the axiom you apply on the right of the derivation (e.g law of associativity, De Morgan's law, etc).

$$[(w \land (x \Rightarrow y)) \lor (x \land (\neg z))] \lor (\neg w)$$

\answerspacefullpage
\additionalanswerspacefullpage

\freespace

\end{document}